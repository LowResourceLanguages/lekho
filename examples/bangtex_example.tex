\documentclass[15pt]{barticle}

\textwidth=12cm
\textheight=18cm
\oddsidemargin=0mm
\evensidemargin=0mm
\tolerance=10000
\renewcommand{\baselinestretch}{1.2}
\begin{document}

This is an example of how to use lekho and bangtex together\footnote{Please read the bantex manuals for details on how to use bangtex itself}. 

Lekho removes the burden of having to read romanised bangla with asterix marks in the old way of typing bangla in bangtex.

To write bangla in a primarily english document use \verb!{\bng xx}! to enclose the bangla part (depicted here as xx), 

The following code\\
\verb!{\bng! 
{\bng এই ভাবে} \verb!{\tt \LaTeX file}! {\bng এ বাংলা এবং ইংরাজি মেলান যায়} \verb!}!

will produce\\
{\bng
এই ভাবে {\tt \LaTeX ~file} এ বাংলা এবং ইংরাজি মেলান যায়
}

To write english in a primarily bangla document start off with \verb!\bng! at the very begining. The whole document is then in bangtex mode. Insert English by typing \verb!\tt{ xx }!, {\tt xx} being the English part.

...\\
\verb!\begin{document}!\\
\verb!\bng!\\
...\\
\verb!{\tt This is the English part}!\\
...\\
\verb!\end{document}!

\framebox{\parbox[t]{10cm}{\it Remember, in {\bng লেখ} hitting ctrl-L will export the current file into a bangtex file. The first time you hit ctrl-L {\bng লেখ} will ask for a file name}}

\noindent The following lines are just a test to see if the {\bng যুক্তাক্ষর} work :\\
{\lbng শবযান্তা\\
চঁাদে\\
কাব্য\\
ক্লাসিক্‌\\
পরিক্ষা\\
স্বরে\\
চিত্‌কার\\
}

\end{document}